\section{Einleitung}\label{chap:Einleitung}

Logistikunternehmen, Busunternehmen oder die Müllabfuhr stehen täglich vor dem Problem, alle geplanten Zielstationen in möglichst kurzer Zeit, mit möglichst geringem Treibstoffverbrauch und unter Berücksichtigung weiterer Nebenbedingungen anzufahren.\newline     
In dieser Arbeit soll das Optimierungsverfahren Ant Colony Optimization (ACO)\nomenclature{\(ACO\)}{Ant Colony Optimization}, das u.A. für die Routenoptimierung eingesetzt wird, vorgestellt werden. \newline
Um Inhalte anschaulich und interaktiv veranschaulichen zu können, wurde hierfür ein Dashboard mittels R-Shiny erstellt. \newline
Ziel dieses Projektberichts ist es, den Quellcode des Dashboards zu erläutern sowie den Aufbau, die Inhalte und die Auswahl der Elemente des R-Shiny-Dashborads zu beschreiben und zu begründen. \newline
Hierbei wird zunächst auf die Struktur des Dashboards eingegangen. Weiterhin wird erläutert, wie allgemeine Informationen zu ACO im Dashboard dargestellt werden. Anschließend wird erläutert, wie die Vorgehensweise des Algorithmus  im R-Shiny Dashboard unter Verwendung interaktiver Elemente visualisiert wird. Weiterhin wird das im Dashboard dargestellte Problem des Handlungsreisenden als Anwendungsgebiet des Algorithmus erläutert und die Umsetzung eines Performance-Vergleichs mit anderen Algorithmen im Dashboard erläutert. 
Zuletzt werden Inhalte und gewonnene Erkenntnisse in einem Fazit zusammengefasst. 