\section{Fazit}\label{chap:Fazit}
Zusammenfassend kann gesagt werden, dass sich ein R-Shiny Dashoard gut eignet, um ein Optimierungsverfahren zu präsentieren und die Herkunft, Wirkungsweise, Anwendungsbeispiele und Performance eines Algorithmus anschaulich darzustellen. Mit relativ einfachen Mitteln kann mit R-Shiny ein ansprechendes und übersichtliches Dashboard erstellt werden, das dem Benutzer einen Überblick über wichtige Aspekte des Algorithmus gibt.\newline
Als besonders sinnvoll erweist sich hierbei die Verwendung interaktiver Elemente. \newline
Durch diese kann dem Benutzer beispielsweise die Möglichkeit gegeben werden, Parametereinstellungen des Algorithmus zu ändern, die sich auf das Ergebnis auswirken, wodurch der Benutzer die Auswirkungen von Parameteränderungen auf das Ergebnis selbst beliebig erproben kann. 
Indem der Benutzer bei der Benutzung des Dashboards aktiv sein kann, erhöht sich folglich sein Interesse und der Lerneffekt.  
Indem auf ein durchgehend schlichtes Design geachtet wird und überfüllte Ansichten vermieden werden, wird zudem erreicht, dass klar verständlich ist, welcher Inhalt relevant ist und der Fokus auf dem Wesentlichen liegt.

